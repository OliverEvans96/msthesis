\chapter{LIGHT MODEL}
\label{chap:light}

Now that we have formulated the distribution of kelp throughout the medium, we introduce the radiative transfer equation, which is used to calculate the light field.

\subsection{Optical Definitions}
One of the most fundamental quantities in optics is radiant flux $\Phi$, which is the has units of energy per time.
The quantity of primary interest in modeling the light field is radiance $L$, which is defined as the radiant flux per steradian per projected surface area perpendicular to the direction of propagation of the beam.
That is,
\begin{equation}
	L = \frac{d^2\Phi}{dA d\omega}
\end{equation}
We must now define a few inherent optical properties (IOPs) which depend only on the medium of propagation.

\subsection{Characteristic Rays}
Consider a fixed position $\vec{x}$ and direction $\vec{\omega}$ such that
$\vec{\omega} \cdot \hat{z} \neq 0$.

% TODO: Just call $\vec{x_0}$ a point, not a function. Call it the projection to the surface.

Let $\vec{l}(\vec{x}, \vec{\omega}, s)$ denote the linear path containing $\vec{x}$
with initial z coordinate given by
\begin{equation}
  z_0 =
   \begin{cases}
    0, & \vec{\omega} \cdot \hat{z} < 0 \\
    \zmax, & \vec{\omega} \cdot \hat{z} > 0
  \end{cases}
\end{equation}

Then,
\begin{equation}
  \vec{l}(\vec{x}, \vec{\omega}, s) = \frac{1}{\tilde{s}} (s\vec{x} + (\tilde{s} - s)\vec{x_0}(\vec{x}, \vec{\omega}))
\end{equation}

where
\begin{equation}
  \vec{x_0}(\vec{x}, \vec{\omega}) = \vec{x} - \tilde{s} \vec{\omega}
\end{equation}
is the origin of the ray, and 

\begin{equation}
  \tilde{s} = \frac{\vec{x} \cdot \hat{z} - z_0}{\vec{\omega} \cdot \hat{z}}
\end{equation}
is the path length from $\vec{x_0}(\vec{x}, \vec{\omega})$ to $\vec{x}$.

\subsection{Colloquial Description}
Denote the radiance at $\vec{x}$ in the direction $\vec{\omega}$ by $L(\vec{x}, \vec{\omega})$.
As light travels along $\vec{l}(\vec{x}, \vec{\omega}, s)$, interaction with the
medium produces three phenomena of interest:
\begin{enumerate}
  \item Radiance is decreased due to absorption.
  \item Radiance is decreased due to scattering out of the path to other
    directions.
  \item Radiance is increased due to scattering into the path from other
      directions.
\end{enumerate}

\subsection{IOPs}
These phenomena are governed by three inherent optical properties (IOPs) of the
medium.
The absorption coefficient $a(\vec{x})$ (units m$^{-1}$) defines the
proportional loss of radiance per unit length.
The scattering coefficient $b$ (units m$^{-1}$), defines the proportional loss
of radiance per unit length, and is assumed to be constant over space.

The volume scattering function (VSF) $\beta(\Delta): [-1, 1] \to \RR^+$ (units sr$^{-1}$) defines the probability of light scattering at any given angle from its source.
Formally, given two directions $\vec{\omega}$ and $\vec{\omega}'$, $\beta(\vec{\omega} \cdot \vec{\omega}')$ is the probability density of light scattering from $\vec{\omega}$ into $\vec{\omega}'$ (or vice-versa).
Of course, since a single direction subtends no solid angle, the probability of scattering occuring exactly from $\vec{\omega}$ to $\vec{\omega}'$ is 0.
Rather, we say that the probability of radiance being scattered from a direction $\omega$ into an element of solid angle $\Omega$ is $\int_\Omega \beta(\vec{\omega} \cdot \vec{\omega}')\, d\vec{\omega}'$.

The VSF is normalized such that
\begin{equation}
  \int_{-1}^1\beta(\Delta)\, d\Delta=\frac{1}{2\pi},
\end{equation}
so that for any $\omega$,
\begin{equation}
  \int_{4\pi}\beta(\vec{\omega}\cdot\vec{\omega}')\, d\vec{\omega}' = 1.
\end{equation}
i.e., the probability of light being scattered to some direction on the unit sphere is 1.

\subsection{Equation of Transfer}
Then, combining these phenomena, the Radiative Transfer equation along
$\vec{l}(\vec{x}, \vec{\omega})$ becomes
\begin{equation}
  \label{eqn:rte1d}
  \frac{dL}{ds}(\vec{l}(\vec{x}, \vec{\omega}, s), \vec{\omega})
  = -(a(\vec{x}) + b)L(\vec{x}, \vec{\omega})
  + b \int_{4\pi} \beta(\vec{\omega}\cdot\vec{\omega}') L(\vec{x})\, d\omega',
\end{equation}
where $\int_{4\pi}$ denotes integration over the unit sphere.

Now, we have
\begin{align*}
  \frac{dL}{ds}(\vec{l}(\vec{x}, \vec{\omega}, s), \vec{\omega})
    &= \frac{d\vec{l}}{ds}(\vec{x}, \vec{\omega}, s) \cdot \nabla L(\vec{x}, \vec{\omega}', \vec{\omega}) \\
    &= \vec{\omega} \cdot \nabla L(\vec{x}, \vec{\omega})
\end{align*}

Then, the general form of the Radiative Transfer Equation is
\begin{equation}
  \vec{\omega} \cdot \nabla L(\vec{x}, \vec{\omega})
  = -(a(\vec{x}) + b)L(\vec{x}, \vec{\omega})
  + b \int_{4\pi} \beta(\vec{\omega}\cdot\vec{\omega}') L(\vec{x}, \vec{\omega}')\, d\omega'
\end{equation}

or, equivalently,
\begin{equation}
  \vec{\omega} \cdot \nabla L(\vec{x}, \vec{\omega})
  + a(\vec{x})L(\vec{x}, \vec{\omega})
  = b \left(
    \int_{4\pi} \beta(\vec{\omega}\cdot\vec{\omega}') L(\vec{x}, \vec{\omega}')\, d\omega'
    - L(\vec{x}, \vec{\omega})
  \right)
\end{equation}

\subsection{Boundary Conditions}

We use periodic boundary conditions in the $x$ and $y$ directions.
\begin{align}
  L\left((\xmin, y, z), \vec{\omega}\right) &= L\left((\xmax, y, z), \vec{\omega}\right) \\
  L\left((x, \ymin, z), \vec{\omega}\right) &= L\left((x, \ymax, z), \vec{\omega}\right)
\end{align}

In the $z$ direction, we specify a spatially uniform downwelling light just
under the surface of the water by a function $f(\vec{\omega})$.
Or if $\zmin>0$, then the radiance at $z=\zmin$ should be specified instead (as opposed to the radiance at the first grid cell center).

Further, we assume that no upwelling light enters the domain from the bottom.
\begin{align}
  L(\vec{x_s}, \vec{\omega}) &= f(\omega) \mbox{ if } \vec{\omega} \cdot \hat{z} > 0\\ 
  L(\vec{x_b}, \vec{\omega}) &= 0 \mbox { if } \vec{\omega} \cdot \hat{z} < 0
\end{align}
 
\section{Irradiance}
Once the radiance $L$ is calculated everywhere, the irradiance is
\begin{equation}
  I(\vec{x}) = \int_{4\pi}L(\vec{x},\vec{\omega})\, d\omega.
\end{equation}
Integrating $I(\vec{x})$, which has units \SI{}{\W/m^2}, over the surface of a frond, produces the power (with units \SI{}{\W}) transmitted to the frond.
This is discussed further in Section \ref{sec:perceived_irrad}
This can be converted to moles of photons (also called einsteins) per second as
\begin{equation}
  \SI{1}{\W\per\m^2} = \SI{4.2}{\micro\mole \,photons\per\second}.
\end{equation}

\section{Asymptotics}
In clear waters where absorption is more important than scattering, an asymptotic expansion can be used whereby the light field is generated through a sequence of discrete scattering events.
\subsection{Substitute asymptotic series}
Taking $b$ to be small, we introduce the asymptotic series
\newcommand{\Lasym}{L_0(\vec{x},\vec{\omega}) + b L_1(\vec{x},\vec{\omega}) + b^2 L_2(\vec{x},\vec{\omega}) + \cdots}
\newcommand{\Lasyms}{L_0(\vec{x_s},\vec{\omega}) + b L_1(\vec{x_s},\vec{\omega}) + b^2 L_2(\vec{x_s},\vec{\omega}) + \cdots}
\newcommand{\Lasymp}{L_0(\vec{x},\vec{\omega}') + b L_1(\vec{x},\vec{\omega}') + b^2 L_2(\vec{x},\vec{\omega}') + \cdots}
\begin{align}
  L(\vec{x},\vec{\omega}) = \Lasym.
\end{align}
Then, substituting the above into the RTE,
\begin{equation}
  \begin{split}
    &\vec{\omega} \cdot \nabla \left[ \Lasym \right] \\
    &+ a(\vec{x}) \left[ \Lasym \right] \\
    &= b\Bigg(
      \int_{4\pi} \beta(\abs{\vec{\omega} - \vec{\omega}'})
      \left[ \Lasymp \right] \, d\vec{\omega}' \\
    &- \left[ \Lasym \right]
    \Bigg)
    \end{split}
\end{equation}
Then, grouping like powers of $b$, we have the decoupled set of equations
\begin{align}
  \vec{\omega} \cdot \nabla L_0(\vec{x}, \vec{\omega}) + a(\vec{x})L_0(\vec{x}) &= 0 \\
  \vec{\omega} \cdot \nabla L_1(\vec{x}, \vec{\omega}) + a(\vec{x})L_1(\vec{x})
  &= \int_{4\pi} \beta(\abs{\vec{\omega} - \vec{\omega}'}) L_0(\vec{x}, \vec{\omega}')\,d\vec{\omega}' - L_0(\vec{x}, \vec{\omega}) \\ 
  \vec{\omega} \cdot \nabla L_2(\vec{x}, \vec{\omega}) + a(\vec{x})L_2(\vec{x})
  &= \int_{4\pi} \beta(\abs{\vec{\omega} - \vec{\omega}'}) L_1(\vec{x}, \vec{\omega}')\,d\vec{\omega}' - L_1(\vec{x}, \vec{\omega}) \\ 
  &\vdots \nonumber
\end{align}

For boundary conditions, let $x_s$ be a point on the surface of the domain.
Then, 
\begin{equation}
  \Lasyms =
  \begin{cases}
    f(\omega), & \hat{z}\cdot\omega > 0 \\
    0, & \mbox{otherwise},
  \end{cases}
\end{equation}
which becomes
\begin{align}
  L_0(\vec{x}, \vec{\omega}) &=
  \begin{cases}
    f(\omega), & \hat{z}\cdot\omega > 0, \\
    0, & \mbox{otherwise},
  \end{cases} \\
  L_1(\vec{x}, \vec{\omega}) &= 0 \\
  L_2(\vec{x}, \vec{\omega}) &= 0. \\
  &\vdots \nonumber
\end{align}

 
\subsection{Rewrite as ODE along ray path}
For all $\vec{x}, \vec{\omega}$, let
\begin{align}
  \tilde{a}(s) &= a(\vec{l}(\vec{x}, \vec{\omega}), s), \\ 
  \frac{du_0}{ds}(s) + \tilde{a}(s) u_0(s) &= 0, u_0(0) = f(\vec{\omega})
\end{align}
Then,
\begin{align}
  u_0(s) = f(\omega) \exp\left(-\int_0^s \tilde{a}(s)\, ds\right), \\
  L_0(\vec{l}(\vec{x}, \vec{\omega},s), \vec{\omega}) = u_0(s)
\end{align}

\begin{align}
  g_n(s) = \int_{4\pi} \beta(\abs{\vec{\omega} - \vec{\omega}'})
  L_{n-1}(\vec{l}(\vec{x}, \vec{\omega'}, s), \vec{\omega}')\,d\vec{\omega}' - L_{n-1}(\vec{l}(\vec{x}, \vec{\omega}, s), \vec{\omega}) \\ 
  \frac{du_n}{ds}(s) + \tilde{a}(s)u_n(s) = g_n(s), u_n(0) = 0
\end{align}

Then,
\begin{align}
  u_n(s) = \int_0^sg_n(s')\exp\left( -\int_{s''}^{s'}\tilde{a}(s'')\,ds'' \right)\, ds' \\
  L_n(\vec{l}(\vec{x}, \vec{\omega}, s), \vec{\omega}) = u_n(s)
\end{align}
