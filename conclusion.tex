\chapter{CONCLUSION}
\label{chap:conclusion}

This thesis presents a model for the light field in a vertical--rope kelp farming operation, emphasizing the effect of the seaweed itself on the overall light field.
A three--dimensional model for the spatial distribution of the seaweed is developed, which then informs the absorption field of the combined seaweed--water medium.
This absorption field is used as a coefficient in the monochromatic radiative transfer equation, which calculates the full light field over all positions and angles in the domain, accounting for both attenuation and scattering in the medium.

Two numerical solution techniques are presented: finite difference and numerical asymptotics.
The finite difference approach is applicable to any type of water, although it can be prohibitively expensive in terms of CPU time, and even more so in terms of memory usage.
On the other hand, the numerical asymptotics algorithm is much computationally cheaper by both measures, though it is only accurate for low--scattering scenarios.
Within the appropriate range of optical properties of the aquatic medium, the accuracy and computation time of the solution can be tuned by choosing the number of terms to include in the asymptotic series.

When compared to simpler one--dimensional models for the light field, the model presented in this thesis is found to predict lower light levels.
This is expected, as the full three--dimensional model considers self--shading due to the kelp in greater detail than the others.
Further, the average irradiance as a function of depth predicted by the three--dimensional model is found to agree fairly well with a simpler model for kelp shading.
However, the average irradiance considers areas of low shading far from the kelp which are irrelevant to photosynthesis.
When the light field is examined only in the regions where kelp is actually growing, much less light predicted.
This indicates that simpler models may be overestimating the amount of light available for photosynthesis, which would, in light--limited situations involving high kelp density or low nutrient concentration, predict unrealistically large overall biomass yields in a time--dependent kelp growth simulation.

\section{Model Summary}
The following is a summary of the primary assumptions used in the formulation of the model.
Each of these assumptions are vast simplifications from the real system, and leave room for future improvement.
\begin{itemize}
  \item All fronds in the population are congruent kites of equal thickness.
  \item Fronds are perfectly flat and horizontal.
  \item Fronds eminate from an infinitely thin, perfectly vertical rope, with no stipe.
  \item Population frond lengths are normally distributed in each depth layer.
  \item Fronds are oriented according to a von Mises distribution whose sharpness is proportional to current velocity and independend of frond length.
  \item Absorption coefficient of the aquatic medium is constant within a depth layer.
  \item All fronds have the same absorptance.
  \item The scattering coefficient is constant everywhere and equal for both kelp and water.
  \item Only the volumetric optical effects, not surface effects, of the kelp are considered.
  \item Frequency dependence is neglected.
\end{itemize}

% \section{Numerical Algorithms}
% \subsection{Choice of Algorithm}
% \subsection{Computational Expense}
%\section{Summary of Results}

\section{Future Work}
Lots of things.
\subsection{Model Improvements}
% Many aspects of the model have room for future improvement.
% The most pressing is probably the development of a model for long--lines, which
% is more popular in practice than the vertical lines studied here.
% Similar techniques can likely be applied, but the details will of course differ.
% 
% One major simplification in the calculation of the kelp model
% is the assumption that the fronds are perfectly horizontal.
% This could be improved in a straightforward way by including some
% probability distribution for the angular elevation as a function of current speed,
% similar to the study performed in \citep{norvik_design_2017}.
% The cost of implementing polar rotation is that depth layers are no longer isolated.
% Rather than integrating the two dimensional length-orientation distribution from
% Section \ref{sec:dist_2d} to calculate the spatial kelp distribution,
% it would be necessary to perform a triple integral which includes the elevation distribution.
% Since frond elevation and azimuathal orientation are both related to current velocity,
% it would likely be impossible to ignore the remarks at the end of \ref{sec:dist_2d}, and the
% assumption of independent distributions would have to be abandoned.
% 
% Of course, real fronds are not rotating planar kites, but have a very dynamic geometry.
% To consider out-of-plane frond bending would require a totally different approach.
% Whether or not any improved description of the seaweed would merit the substantial work is unclear.
\begin{itemize}
  \item Long lines.
  \item non--horizontal fronds
  \item determine $\eta$ experimentally
  \item size--dependent frond rotation, Bayes' thm.
  \item population--varying kelp params: $f_t$, $A_k$.
  \item non--uniform thickness within a frond
  \item Frequency dependence, inelastic scattering
  \item Kelp scattering coefficient.
\end{itemize}

\subsection{Numerical Improvements}
\begin{itemize}
  \item MPI LIS
  \item grid--local num. asymptotics calculations
  \item Alternate angular grids (Lebedev, ``spherical t--design'' --- Beentjes)
  \item FEM / Monte Carlo
  \item Other 3D RTE methods
\end{itemize}

\subsection{Application}
\begin{itemize}
  \item Use in actual simulations
  \item Determine optimal/maximum rope density as in Broch Modelling 2013.
  \item Compare long lines to vertical lines
  \item Model IOPs as functions of nutrients, etc. to determine viability of IOPs
\end{itemize}


