\chapter{CONCLUSION}
\label{chap:conclusion}

% We present a probabilistic model for the spatial distribution of kelp, and develop a first--principles model for the light field, considering absorption and scattering due to the water and kelp.
% A full finite difference solution is presented, and an asymptotic approximation based on discrete scattering events is subsequently developed.
% The asymptotic approximation is shown to converge to the finite difference solution in cases where the absorption coefficient is the same order
% of magnitude as the scattering coefficient or larger.
% Otherwise, the solution diverges.

% TODO: Include summary of results
\section{Model Summary}
\subsection{List of Assumptions}
\begin{itemize}
  \item kite fronds
  \item uniform frond thickness
  \item horizontal fronds
  \item vertical, thin rope => blur
  \item ignore stipe
  \item normally distributed lengths at each depth
  \item VM distributed angles at each depth
  \item Angular dist. sharpness propto current speed
  \item Independent length/angle dists
  \item Horizontally uniform $a_w$
  \item $A_k$ same for all fronds
  \item $b$ const. everywhere (kelp and water)
  \item no surface optics, only volumetric
  \item Monochromatic light
\end{itemize}

\subsection{Physical Parameters}

\section{Numerical Algorithms}
\subsection{Choice of Algorithm}
\subsection{Computational Expense}

\section{Summary of Results}

\section{Future Work}
Lots of things.
\subsection{Model Improvements}
% Many aspects of the model have room for future improvement.
% The most pressing is probably the development of a model for long--lines, which
% is more popular in practice than the vertical lines studied here.
% Similar techniques can likely be applied, but the details will of course differ.
% 
% One major simplification in the calculation of the kelp model
% is the assumption that the fronds are perfectly horizontal.
% This could be improved in a straightforward way by including some
% probability distribution for the angular elevation as a function of current speed,
% similar to the study performed in \citep{norvik_design_2017}.
% The cost of implementing polar rotation is that depth layers are no longer isolated.
% Rather than integrating the two dimensional length-orientation distribution from
% Section \ref{sec:dist_2d} to calculate the spatial kelp distribution,
% it would be necessary to perform a triple integral which includes the elevation distribution.
% Since frond elevation and azimuathal orientation are both related to current velocity,
% it would likely be impossible to ignore the remarks at the end of \ref{sec:dist_2d}, and the
% assumption of independent distributions would have to be abandoned.
% 
% Of course, real fronds are not rotating planar kites, but have a very dynamic geometry.
% To consider out-of-plane frond bending would require a totally different approach.
% Whether or not any improved description of the seaweed would merit the substantial work is unclear.
\begin{itemize}
  \item Long lines.
  \item non--horizontal fronds
  \item determine $\eta$ experimentally
  \item size--dependent frond rotation, Bayes' thm.
  \item population--varying kelp params: $f_t$, $A_k$.
  \item non--uniform thickness within a frond
  \item Frequency dependence, inelastic scattering
\end{itemize}

\subsection{Numerical Improvements}
\begin{itemize}
  \item MPI LIS
  \item grid--local num. asymptotics calculations
  \item Alternate angular grids (Lebedev, ``spherical t--design'' --- Beentjes)
  \item FEM / Monte Carlo
  \item Other 3D RTE methods
\end{itemize}

\subsection{Application}
\begin{itemize}
  \item Use in actual simulations
  \item Determine optimal/maximum rope density as in Broch Modelling 2013.
  \item Compare long lines to vertical lines
  \item Model IOPs as functions of nutrients, etc. to determine viability of IOPs
\end{itemize}


