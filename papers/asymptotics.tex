
\documentclass[10pt]{article}

\usepackage{graphicx}
\usepackage{amsmath}
\usepackage{amssymb}
\usepackage{mathtools}
\usepackage{etoolbox}
\usepackage{booktabs}
\usepackage[parfill]{parskip}
\usepackage[numbers]{natbib}
\usepackage{float}
\usepackage{graphicx}
\usepackage{geometry}
\usepackage{multicol}
\usepackage{caption}
\usepackage{url}
% \usepackage[T1,T2A]{fontenc}
% \usepackage{lmodern}
% \usepackage[utf8]{inputenc}
% \usepackage[russian,english]{babel}
% 
% \renewcommand{\thesection}{\bfseries\arabic{section}}
% \renewcommand{\thesubsection}{\bfseries\arabic{subsection}}
% \renewcommand{\thesubsubsection}{\bfseries\arabic{subsubsection}}

\newcommand\mgin{0.5in}
\geometry{
    left=\mgin,
    right=\mgin,
    bottom=\mgin,
    top=\mgin
}

% Set path to import figures from
\graphicspath{{../../img/sparsity/}}

% Place converted graphics in current directory
\usepackage[outdir=./]{epstopdf}

% Define multicolumn figure-like environment
% from http://tex.stackexchange.com/questions/12262/multicol-and-figures
\newenvironment{mcfig}
    {\par\medskip\noindent\minipage{\linewidth}}
    {\endminipage\par\medskip}

% Define error function for math mode
\newcommand{\erf}{\mbox{erf}}
% Sign function
\newcommand{\sign}{\mbox{sign}}
% Natural numbers
\newcommand\NN{\mathbb{N}}
% Real numbers
\newcommand\RR{\mathbb{R}}
% Complex numbers
\newcommand\CC{\mathbb{C}}
% Curly B for basis
\newcommand\BB{\mathcal{B}}
% Curly D for diagonal dominance quantity
\newcommand\DD{\mathcal{D}}
\newcommand\QQ{\mathcal{Q}}
% Norm
\newcommand\norm[1]{\left\lVert #1 \right\rVert}
% Uniform Norm
\newcommand\unorm[1]{\left\lVert #1 \right\rVert_\infty}
% Inner Product
\newcommand\ip[1]{\left\langle #1 \right\rangle}
% Absolute value
\newcommand\abs[1]{\left| #1 \right|}
% Complex Conjugate
\newcommand\conj\overline
% Partial derivative
\newcommand\pd[2]{\frac{\partial #1}{\partial #2}}
% Iteration superscript w/ parentheses
\newcommand{\iter}[1]{^{(#1)}}
% Disable paragraph indentation
\setlength{\parindent}{0pt}
% End of proof
\newcommand\qed{\hfill$\blacksquare$\hspace{0.5in}}

% Number this equation
\newcommand\eqnum{\addtocounter{equation}{1}\tag{\theequation}}

% arara: pdflatex
\begin{document}

\section{Introduction}
\subsection{Coordinate System}
We use a downward-facing right-handed Cartesian coordinate system.
For spherical coordinates, we denote polar declenation from the positive $z$
axis by $\phi$ and azimuthal angle from the positive x-axis towards the positive
y-axis by $\theta$.
\subsection{Radiative Transfer Equation}
\begin{equation}
  \label{eq:RTE}
  \vec{\omega} \cdot \nabla L(\vec{r},\vec{\omega})
  = -(a(\vec{r}) + b(\vec{r}) + b\int_{4\pi}
  \beta(\vec{\omega} \cdot \vec{\omega'})
  L(\vec{r},\vec{\omega'}) \, d\vec{\omega'})
\end{equation}

\subsection{Boundary conditions}
Downwelling light:
\begin{equation}
  L(x,y,0, \vec{\omega}) = f(\vec{\omega})
\end{equation}

Upwelling light:
\begin{equation}
  L(x,y,M, \vec{\omega}) = 0
\end{equation}

\subsection{All angles are coupled by scattering}
A major disadvantage of this formulation is that at each point in space,
radiances in all angles are coupled. By asymptotic expansion, the solution can
be broken up into independent scattering events, in which angles are decoupled
by replacing the scattering integral with an integral over the radiance from the
previous scattering event, which is already known.

\section{Nondimensionalization}
We nondimensionalize following Chandrasekhar.
\subsection{Assumptions}
We assume that the scattering coefficient is constant over space; that the
primary difference in the optical effects of kelp and water is absorption, not
scattering.

\subsection{Scattering constant (same for kelp/water)}
\subsection{Table of variables}
\subsection{Rescale space, time}

\section{Limitations of Discrete Ordinates}
\subsection{Memory}
GMRES is very memory-intensive.
\subsection{CPU}
It's very CPU-intensive as well.
\subsection{GMRES unreliable}
It also might never converge!
\subsection{Also, need an initial guess for GMRES}
Especially if we start from zero.
  
\section{Asymptotics}
\subsection{Solve each angular problem independently}
\subsection{Relatively computationally cheap}
\subsection{Much lower memory cost}
\subsection{Known number of operations}
\subsection{Low and high accuracy available}
  
\section{Mathematical Procedure}
\subsection{Substitute asymptotic series}
\newcommand{\Lasym}{L_0(\vec{r},\vec{\omega}) + b L_1(\vec{r},\vec{\omega}) + b^2 L_2(\vec{r},\vec{\omega}) + \cdots}
\newcommand{\Lasymp}{L_0(\vec{r},\vec{\omega'}) + b L_1(\vec{r},\vec{\omega'}) + b^2 L_2(\vec{r},\vec{\omega'}) + \cdots}
\begin{equation}
  L(\vec{r},\vec{\omega}) = \Lasym
\end{equation}

\begin{equation}
  \vec{\omega} \cdot \nabla \left[ \Lasym \right]
  = -(a(\vec{r}) + b(\vec{r})) + b\int_{4\pi}
  \beta(\vec{\omega} \cdot \vec{\omega'})
  \left[ \Lasymp \right] \, d\vec{\omega'}
\end{equation}
 
\subsection{Group like powers of b}
\subsection{Boundary conditions}
\subsection{Rewrite as ODE along ray path}
\subsection{Solve ODE as 1st order linear via I.F.}
  
\section{Numerical Implementation}
\subsection{Discrete grid}
We choose a uniform rectangular spatial grid dividing each dimension into $n_x$,
$n_y$, and $n_z$ grid points respectively.
We also use a uniformly spaced angular grid.

\subsection{Numerical integration}
We use the trapezoid rule for numerical integration as it allows for even or odd
numbers of points, and places no restriction on the symmetry of the grid.
\subsection{Storing pole values}
We store pole values in the $(1,1)$ and $(1,n_\phi)$ positions.
\subsection{Loop rolling}
\subsection{Scattering integral}
We do not include the current direction in the scattering integral.
  
\end{document}