

\documentclass[10pt]{article}

\usepackage{graphicx}
\usepackage{amsmath}
\usepackage{amssymb}
\usepackage{mathtools}
\usepackage{etoolbox}
\usepackage{booktabs}
\usepackage[parfill]{parskip}
\usepackage[numbers]{natbib}
\usepackage{float}
\usepackage{graphicx}
\usepackage{geometry}
\usepackage{multicol}
\usepackage{caption}
\usepackage{url}
% \usepackage[T1,T2A]{fontenc}
% \usepackage{lmodern}
% \usepackage[utf8]{inputenc}
% \usepackage[russian,english]{babel}
% 
% \renewcommand{\thesection}{\bfseries\arabic{section}}
% \renewcommand{\thesubsection}{\bfseries\arabic{subsection}}
% \renewcommand{\thesubsubsection}{\bfseries\arabic{subsubsection}}

\newcommand\mgin{0.5in}
\geometry{
    left=\mgin,
    right=\mgin,
    bottom=\mgin,
    top=\mgin
}

% Set path to import figures from
\graphicspath{{../../img/sparsity/}}

% Place converted graphics in current directory
\usepackage[outdir=./]{epstopdf}

% Define multicolumn figure-like environment
% from http://tex.stackexchange.com/questions/12262/multicol-and-figures
\newenvironment{mcfig}
    {\par\medskip\noindent\minipage{\linewidth}}
    {\endminipage\par\medskip}

% Define error function for math mode
\newcommand{\erf}{\mbox{erf}}
% Sign function
\newcommand{\sign}{\mbox{sign}}
% Natural numbers
\newcommand\NN{\mathbb{N}}
% Real numbers
\newcommand\RR{\mathbb{R}}
% Complex numbers
\newcommand\CC{\mathbb{C}}
% Curly B for basis
\newcommand\BB{\mathcal{B}}
% Curly D for diagonal dominance quantity
\newcommand\DD{\mathcal{D}}
\newcommand\QQ{\mathcal{Q}}
% Norm
\newcommand\norm[1]{\left\lVert #1 \right\rVert}
% Uniform Norm
\newcommand\unorm[1]{\left\lVert #1 \right\rVert_\infty}
% Inner Product
\newcommand\ip[1]{\left\langle #1 \right\rangle}
% Absolute value
\newcommand\abs[1]{\left| #1 \right|}
% Complex Conjugate
\newcommand\conj\overline
% Partial derivative
\newcommand\pd[2]{\frac{\partial #1}{\partial #2}}
% Iteration superscript w/ parentheses
\newcommand{\iter}[1]{^{(#1)}}
% Disable paragraph indentation
\setlength{\parindent}{0pt}
% End of proof
\newcommand\qed{\hfill$\blacksquare$\hspace{0.5in}}

% Number this equation
\newcommand\eqnum{\addtocounter{equation}{1}\tag{\theequation}}

% arara: pdflatex
\begin{document}

\section{Introduction}
An accurate representation of the underwater light field is essential in
estimating production rates in aquatic biological models. While it may suffice
to use a simple model for downward attenuation in certain circumstances, such as
modeling the growth of phytoplankton, some situations which are characterized by
significant spatial variance and angular asymmetries such as macroalgae
populations may require a more sophisticated optical model in order to account
for self-shading, especially considering the dynamic spatial configuration of
the kelp in an environment with significantly time-varying water velocity.

On the other hand, ocean-scale hydrodynamical models such as SINMOD generally require
computations over large spatial domains, which significantly limit the share of
computational resources which can be devoted to any particular aspect of the
model. Light models in particular can be very computationally intensive due to
the large number of dimensions involved ignoring symmetries (three spatial, two
angular). Verily, no benefit will be derived from a realistic model which
computational scientists reject for it's computational intensity.

Our objective, then, is to devise a computational technique which accurately
depicts the underwater radiation field, while providing a significant
improvement over the model of additive vertical exponential decay, which is
prevelant throughout biological models. The approches considered here are
sourced from the theory of radiative transfer, a macroscopic first-principles
formalism. While this approach is widely used in stellar astrophysics, its use
in oceanography is scarce and has been largely pioneered by Curtis Mobley in the
past half-century.

In particular, we consider the techniques of discrete ordinates, which provides
high accuracy at high computational cost; analytical solutions from horizontal
homogenization, which is the usual over-simplification used in biology; and the
asymptotic techniques of dividing computation into distinct scattering
events, which provides excellent approximations with reasonable expense, and
whose economy can be variable increased depending on the acceptible error in
the solution.
  
\end{document}
