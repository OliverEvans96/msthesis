\chapter{KELP MODEL}
\label{label:kelp}

\section{Physical Setup}
Being a salt water species, macroalgae cultivation occurs primarily in the ocean, with the exception of the initial stage of growth, where microscopic kelp spores are inoculated onto a thread in a small laboratory pool.
This thread is then wrapped around a large rope, which is placed in the ocean and generally suspended by buoys in one of two configurations: horizontal or vertical.
Thus far, I am primarily concerned with modeling the vertical rope case, in which the kelp plants extend radially outward from the rope in all directions, which are made up of a single frond (leaf), stipe (stem) and holdfast (root structure).
We consider a rectangular grid of such vertical ropes. 
Plants extending from each rope will shade both themselves and their neighbors
to varying degrees based on the depth of the kelp, the rope spacing, the angle
of incident light on the surface and the nature of scattering in the water.
In addition, light will be naturally absorbed by the water to varying degrees as determined by the clarity of the water.

\begin{figure}[H]
	\centering
	\includegraphics[width=3.5in]{kelp_array}
	\captionof{figure}{$4\times 4$ array of vertical kelp ropes}
\end{figure}
\subsection{Coordinate System}

Consider the rectangular domain
\begin{align*}
  \xmin &\leq x \leq \xmax, \\
  \ymin &\leq y \leq \ymax, \\
  \zmin &\leq z \leq \zmax.
\end{align*}
For all three dimensional analysis, we use the absolute coordinate system defined in figure \ref{fig:3dcoords}.
In the following sections, it is necessary to convert between Cartesian and spherical coordinates, which we do using the relations
\begin{align}
	\begin{split}
		x & = r\sin\phi\cos\theta, \\
		y & = r\sin\phi\sin\theta, \\
		z & = r\cos\phi. \\
	\label{eqn:coords}
	\end{split}
\end{align}

Therefore, for some function $f(x,y,z)$, we can write its derivative along a path in spherical coordinates in terms of Cartesian coordinates using the chain rule.
\begin{equation}
	\frac{\partial f}{\partial r} 
	=\frac{\partial f}{\partial x}\frac{\partial x}{\partial r} 
	+ \frac{\partial f}{\partial y}\frac{\partial y}{\partial r} 
	+ \frac{\partial f}{\partial z}\frac{\partial z}{\partial r}
\end{equation}
Then, calculating derivatives from \eqref{eqn:coords} yields
\begin{equation}
	\frac{\partial f}{\partial r} 
	=\frac{\partial f}{\partial x}\sin\phi\cos\theta
	+ \frac{\partial f}{\partial y}\sin\phi\sin\theta
	+ \frac{\partial f}{\partial z}\cos\phi.
	\label{eqn:partials}
\end{equation}
\begin{figure}[H]
	\centering
	\includegraphics[width=3in]{3d_coords}
	\caption{Downward-facing right-handed coordinate system with radial distance $r$ from the origin, distance $s$ from the $z$ axis, zenith angle $\phi$ and azimuthal angle $\theta$}
	\label{fig:3dcoords}
\end{figure}


\section{Kelp Model}

\subsection{Frond shape}
\label{sec:shape}

% \begin{figure}[h]
%   \centering
%   \includegraphics[width=0.7\textwidth]{kelp_photo/kite}
%   \caption{Kite-like shape of \textit{Saccharina Latissima}}
% \end{figure}

\begin{figure}[h]
	\centering
  \includegraphics[width=1.2in]{kelp_photo/kite}
  %TODO: Cite this?
  \qquad
	\includegraphics[width=2in]{frond}
	\captionof{figure}{Simplified kite-shaped frond}
	\label{fig:frond}
\end{figure}

We assume the frond is a kite with length $l$ from base to tip, and width $w$ from left to right.
 %In figure \ref{fig:frond}, the base is shown at the bottom and the tip is shown at the top.
 The shortest distance from the base to the diagonal connecting the left and right corners is called $f_a$, and the shortest distance from that diagonal to the tip is called $f_b$.
 We have
 \begin{equation}
	 f_a + f_b = l
 \end{equation}
When considering a whole population with varying sizes, it is more convenient to specify ratios than absolute lengths.
Let the following ratios be defined.
\begin{align}
	f_r &= \frac{l}{w} \\
	f_s &= \frac{f_a}{f_b}
\end{align}
These ratios are assumed to be consistent among the entire population, making all fronds geometrically similar.
With these definitions, the shape of the frond can be fully specified by $l$, $f_r$, and $f_s$.
It is possible, then, to redefine $w$, $f_a$ and $f_b$ as follows from the preceding formulas.

\begin{align}
	w &= \frac{l}{f_r} \\
	f_a &= \frac{lf_s}{1+f_s} \\
	f_b &= \frac{l}{1+f_s}
\end{align}

The angle $\alpha$, half of the angle at the base corner, is also important in our analysis.
Using the above equations,
\begin{equation}
	\alpha = \tan^{-1}\left(\frac{2f_rf_s}{1+f_s}\right)
\end{equation}

The area of the frond is given by
\begin{equation}
  A = \frac{lw}{2} = \frac{l^2}{2f_r}.
\end{equation}

Likewise, if the area is known, then the length is
\begin{equation}
  l = \sqrt{2Af_r}
  \label{eqn:length-from-area}
\end{equation}

\subsection{Length and angle distributions}
\label{sec:angle_dist}
We assume that frond lengths are normally distributed with mean $\mu_l$ and standard deviation $\sigma_l$.
We assume the frond angle varies according to the von Mises distribution, which is the periodic analogue of the normal distribution, defined on $[-\pi,\pi]$ rather than $(-\infty,\infty)$.
The von Mises distribution has two parameters, $\mu$ and $\kappa$, which shift and sharpen its peak respectively, as shown in Figure \ref{fig:vonmises}.
$\kappa$ can be considered analogous to $1/\sigma$ in the normal distribution.
Here, we use $\mu = \theta_w$ and $\kappa = v_w$.
That is, in the case of zero current velocity, the frond angles are be distributed uniformly, while as current velocity increases, they become increasingly likely to be pointing in the direction of the current.
Note that $\theta_w$ and $v_w$ vary over depth.

The PDF for this distribution is
\begin{equation}
	P_{\theta_f}(\theta_f) = \frac{\exp\left(v_w\cos(\theta_f-v_w)\right)}{2\pi I_0(v_w)}
\end{equation}
where $I_0(x)$ is the modified Bessel function of the first kind of order 0.
Notice that unlike the normal distribution, the von Mises distribution approaches a \textit{non-zero} uniform distribution as $\kappa$ approaches 0.
\begin{equation}
	\displaystyle \lim_{v_w \to 0}P_{\theta_f}(\theta_f) = \frac{1}{2\pi} \;\forall\, \theta_f \in [-\pi,\pi]
\end{equation}

\begin{figure}[h]
	\centering
	\includegraphics[width=.75\linewidth]{vonmises_2}
	\captionof{figure}{von Mises distribution for a variety of parameters}
	\label{fig:vonmises}
\end{figure}

\subsection{Combined 2D length-angle distribution}
\label{sec:2d_dist}
The previous two distributions can reasonably be assumed to be independent of one another. That is, the angle of the frond does not depend on the length, or vice versa.
Therefore, the probability of a frond simultaneously having a given frond length and angle is the product of their individual probabilities.

Given independent events $A$ and $B$,
\begin{equation}
	\label{eq:ind_prob}
	P(A \cap B) = P(A)P(B)
\end{equation}
Then the probability of frond length $l$ and frond angle $\theta_f$ coinciding is 
\begin{equation}
	P_{2D}(\theta_f,l) = P_{\theta_f}(\theta_f) \cdot L(l)
\end{equation}
A contour plot of this 2D distribution for a specific set of parameters is shown in figure \ref{fig:dist_2d}, where probability is represented by color in the 2D plane.
Darker green represents higher probability, while lighter beige represents lower probability.
In figure \ref{fig:kelp_sample}, 50 samples are drawn from this distribution and plotted.

It is important to note that if $P_{\theta_f}$ were dependent on $l$, the above definition of $P_{2D}$ would no longer be valid.
For example, it might be more realistic to say that larger fronds are less likely to bend towards the direction of the current.
In this case, \eqref{eq:ind_prob} would no longer hold, and it would be necessary to use the following more general relation.
\begin{equation}
	P(A \cap B) = P(A)P(B|A) = P(B)P(B|A)
\end{equation}
This is currently not taken into consideration in this model.

\begin{figure}[h]
	\centering
	\includegraphics[width=.75\linewidth]{prob_2d}
	\vspace{-3em}
	\captionof{figure}{2D length-angle probability distribution with $\theta_w=2\pi/3,v_w=1$}
	\label{fig:dist_2d}
\end{figure}

\begin{figure}[h]
	\centering
	\includegraphics[width=.75\linewidth]{kelp_sample}
	\vspace{-2em}
	\captionof{figure}{A sample of 50 kelp fronds with length and angle picked from the distribution above with $f_s=0.5$ and $f_r=2$.}
	\label{fig:kelp_sample}
\end{figure}

