\begin{abstract}
  A mathematical model is developed to describe the light field in
  vertical line seaweed cultivation to determine the
  degree to which the seaweed shades itself and limits the
  amount of light available for photosynthesis.
  A probabilistic description of the spatial distribution of kelp
  is formulated using simplifying assumptions about frond geometry and orientation.
  An integro-partial differential equation called the  radiative transfer equation
  is used to describe the light field as a function of position and angle.
  A finite difference solution is implemented, providing robustness and accuracy
  at the cost of large CPU and memory requirements, and
  a less computationally intensive asymptotic approximation is explored for the case of low
  scattering.
  Conditions for applicability of the asymptotic approximation are discussed,
  and depth-dependent light availability is compared to the predictions of simpler light models.
  The 3D model of this thesis is found to predict significantly lower light levels than the simpler 1D models,
  especially in regions of high kelp density where a precise description of self-shading is most important.
\end{abstract}
