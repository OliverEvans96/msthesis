\chapter{NUMERICAL SOLUTION}
\label{chap:numerical}

In this chapter, the mathematical details involved in the numerical solution of the previously described equations are presented.
It is assumed that this model is run in conjunction with a model describing the growth of kelp over its life cycle, which calls this light model periodically to update the light field.

\section{Super-Individuals}
\label{sec:si}

Rather than model each kelp frond, a subset of the population, called super-individuals, are modeled explicitly, and are considered to represent many identical individuals, as in \citep{scheffer_super-individuals_1994}.
Specifically, at each depth $k$, there are $n$ super-individuals, indexed by $i$.
Super-individual $i$ has a frond area $A_{ki}$ and represents $n_{ki}$ individual fronds.

From \eqref{eqn:length-from-area}, the frond length of the super-individual is $l_{ki} = \sqrt{2A_{ki}f_r}$.
Given the super-individual data, we calculate the mean $\mu$ and standard deviation $\sigma$ frond
lengths using the formulas
\begin{align}
  \mu_k &= \frac{\ds \sum_{i=1}^N l_{ki}}{\ds \sum_{i=1}^N n_{ki}},
  \label{eqn:si_mean} \\
  \sigma_k &= \frac{\ds \sum_{i=1}^N \left( l_{ki} - \mu_k \right)^2}{\ds \sum_{i=1}^N n_{ki}}.
  \label{eqn:si_std}
\end{align}
We then assume that frond lengths are normally distributed in each depth layer
with mean $\mu_k$ and standard deviation $\sigma_k$.

\section{Discrete Grid}
\label{sec:grid}

The following is a description of the spatial-angular grid used in the numerical implementation of this model.
It is assumed that all simulated quantities are constant over the interior of a grid cell.
Other legitimate choices of grids exists; this one was chosen for its relative simplicity.

The domain of the radiative transfer equation is embedded in five dimensions: three spatial ($x$, $y$, and $z$) and two angular (azimuthal $\theta$ and polar $\phi$).
The number of grid cells in each dimension are denoted by $n_x$, $n_y$, $n_z$,
$n_\theta$, and $n_\phi$, with uniform spacings $dx$, $dy$, $dz$, $d\theta$, and
$d\phi$ between adjacent grid points.

The following indices are assigned to each dimension:
\begin{align*}
  x &\to i \\
  y &\to j \\
  z &\to k \\
  \theta &\to l \\
  \phi &\to m
\end{align*}

It is convenient, however, to use a single index $p$ to refer to directions $\vec{\omega}$ rather than referring to $\theta$ and $\phi$ separately.
Then, the center of a generic grid cell will be denoted as
$(x_i, y_j, z_k, \vec{\omega}_p)$, and the boundaries between adjacent grid cells
will be referred to as \textit{edges}.
One-indexing is employed throughout this document.

\begin{figure}[H]
  \centering
  \includegraphics[width=8cm]{spatialgrid.pdf}
  \caption{Spatial grid}
  \label{fig:spatial_grid}
\end{figure}

Each spatial grid cell is the Cartesian product of $x$, $y$, and $z$ intervals of width $dx$, $dy$, and $dz$ respectively.
The three-dimensional interval centered at $(x_i, y_j, z_k)$ is denoted $X_{ijk}$, and has volume $\abs{X_{ijk}}=dx\,dy\,dz$.
Also, note that no grid center is located on the plane $z=0$; the surface radiance boundary condition is treated separately.

\begin{figure}[H]
  \centering
  \includegraphics[width=8cm]{angulargrid.pdf}
  \caption{Angular grid at each point in space}
  \label{fig:angular_grid}
\end{figure}

As shown in Figure \ref{fig:angular_grid}, $\phi=0$ and $\phi=\pi$, called
the north ($+z$) and south ($-z$) poles respectively, are treated separately from other angular grid cells.
A generic interior angular grid cell centered at $\vec{\omega}_p$ is the Cartesian product of an azimuthal interval of width $d\theta$ and a polar interval of width $d\theta$.
However, two pole cells are the Cartesian product of a polar interval of width $d\phi/2$ and the full azimuthal domain, $[0, 2\pi)$.

With this configuration, the total number of angles considered is $\nomega = n_\theta(n_\phi-2)+2$.
Then, cells are indexed by $p=1,\ldots,n_{\vec{\omega}}$ and are ordered such that
$p=1$ and $p=n_{\vec{\omega}}$ refer to the north and south poles respectively,
$p\leq\nomega/2$ refers to the northern hemisphere, and $p>\nomega/2$ refers to the southern hemisphere.
Further, the symbol $\Omega_p$ is used to refer to the two dimension angular interval centered at $\omega_p$.
The solid angle subtended by $\Omega_p$ is denoted $\abs{\Omega_p}$.
Refer to Appendix \ref{chap:grid_details} for a more rigorous discussion of the discrete spatial-angular grid.

\section{Quadrature Rules}
Since it is assumed that all quantities are constant within a spatial-angular grid cell,
the midpoint rule is employed for both spatial and angular integration.
Presented here is a basic derivation of the formulas for integration in the spatial-angular grid.
Further details are found in Appendix \ref{chap:ray_tracing}.

\subsection{Spatial Quadrature}
Define the \textit{spatial characteristic function} as
\begin{equation*}
  \mathcal{X}^X_{ijk}(\vec{x}) = \begin{cases}
    1, & \vec{x} \in X_{ijk} \\
    0, & \mbox{otherwise}.
  \end{cases}
\end{equation*}
The double integral of a function $f(\vec{x})$ over a depth layer $k$ is approximated as
\begin{align*}
  \int_\xmin^\xmax\int_\ymin^\ymax f(x, y, z_k)\, dy\, dx &\approx \int_\xmin^\xmax \int_\ymin^\ymax \sum_{i=1}^{n_x}\sum_{j=1}^{n_y} \mathcal{X}^X_{ijk}(x,y,z_k) f(x_i, y_j, z_k)\, dy\, dx \\
  &= \sum_{i=1}^{n_x}\sum_{j=1}^{n_y} f(x_i, y_j, z_k) \int_\xmin^\xmax \int_\ymin^\ymax \mathcal{X}^X_{ijk}(x,y,z_k) \, dy\, dx \\
  &= \sum_{i=1}^{n_x}\sum_{j=1}^{n_y} \abs{X_{ijk}} f(x_i, y_j, z_k) \\
  &= dx\, dy\, dx\, \sum_{i=1}^{n_x}\sum_{j=1}^{n_y} f(x_i, y_j, z_k).
\end{align*}
The path integral of $f(\vec{x})$ over a path $\vec{l}(s)$ from $s=0$ to $s=\tilde{s}$ is
\begin{align*}
  \int_0^{\tilde{s}} f(\vec{l}(s))\, ds &= \sum_{i=1}^{n_x}\sum_{j=1}^{n_y}\sum_{k=1}^{n_z} f(x_i, y_j, z_k)\, ds_{ijk},
\end{align*}
where $ds_{ijk}$ is the total path distance of $\vec{l}(s)$ through $X_{ijk}$.
Full details of the path integral algorithm for the case of straight line paths are found in Appendix \ref{chap:ray_tracing}.

\subsection{Angular Quadrature}
Define the \textit{angular characteristic function} as
\begin{equation*}
  \mathcal{X}^\Omega_p(\vec{\omega}) = \begin{cases}
    1, & \vec{\omega} \in \Omega_p \\
    0, & \mbox{otherwise}.
  \end{cases}
\end{equation*}


Then, the integral of a function $f(\vec{\omega})$ is approximated as
\begin{align*}
  \int_{4\pi} f(\vec{\omega})\, d\vec{\omega} &\approx \int_{4\pi} \sum_{p=1}^\nomega f(\vec{\omega}_p) \mathcal{X}^\Omega_p(\vec{\omega})\, d\vec{\omega} \\
  &= \sum_{p=1}^\nomega f(\vec{\omega}_p) \int_{4\pi} \mathcal{X}^\Omega_p(\vec{\omega})\, d\vec{\omega} \\
  &= \sum_{p=1}^\nomega f(\vec{\omega}_p) \int_{\Omega_p} d\vec{\omega} \\
  &= \sum_{p=1}^\nomega f(\vec{\omega}_p) \abs{\Omega_p}.
\end{align*}

\section{Numerical Asymptotics}
Presented here are details of the evaluation of the asymptotic approximations \eqref{eqn:asymptotics_soln_0} and \eqref{eqn:asymptotics_soln_n} to the raditiave transfer equation \eqref{eqn:rte}.

\subsection{Scattering Integral}

Specifically, the amount of light scattered between angular grid cells is found by integrating $\beta$ as follows.
Consider two angular grid cells, $\Omega$ and $\Omega'$.
Since $\beta(\vec{\omega}\cdot\vec{\omega}')$ is the probability density of scattering between $\vec{\omega}$ and $\vec{\omega}'$, the average probability density of scattering from $\vec{\omega} \in \Omega$ to $\vec{\omega}' \in \Omega'$ (or vice versa) is
\begin{equation*}
  \beta_{pp'} = \frac{1}{\abs{\Omega}\abs{\Omega'}} \int_\Omega\int_{\Omega'}\beta(\vec{\omega}\cdot\vec{\omega}')\, d\vec{\omega'}\, d\vec{\omega}.
\end{equation*}
Denote the radiance at $(x_i, y_j, z_k, \vec{\omega}_p)$ by $L_{ijkp}$.
Then, the total radiance scattered into $\Omega_p$ from $\Omega_{p'}$ is
\begin{align*}
  \int_{\Omega}\int_{\Omega'}\beta(\vec{\omega} \cdot \vec{\omega}')L(\vec{x},\vec{\omega}')\, d\vec{\omega}'\, d\vec{\omega}
  &= L_{ijkp'} \int_\Omega\int_{\Omega_{p'}} \beta(\vec{\omega} \cdot \vec{\omega}')\, d\vec{\omega}'\, d\vec{\omega} \\
  &= \beta_{pp'}\abs{\Omega}\abs{\Omega'}L_{ijkp'}.
\end{align*}
Hence, the average radiance scattered from $\Omega_{p'}$ into some $\vec{\omega} \in \Omega_p$ is $\beta_{pp'}\abs{\Omega'}L_{ijkp'}$.
Therefore, the radiance gain due to scattering into $\vec{\omega}_p$ from all other angles is
\begin{equation}
  \int_{4\pi}\beta(\vec{\omega_p}\cdot\vec{\omega_{p'}})L(\vec{x}, \vec{\omega}')\, d\vec{\omega} \approx \sum_{p=1}^\nomega \beta_{pp'}\abs{\Omega'}L_{ijkp}.
  \label{eqn:scatter_integral}
\end{equation}

\subsection{Ray Integral}
Given a position $\vec{x}$ and direction $\vec{\omega}$, a path through the discrete grid can be constructed using the ray tracing algorithm described in Appendix \ref{chap:ray_tracing}. 
Let $\nu=1,\ldots,N\nobreak-\nobreak1$ index the spatial grid cells traversed by the ray, and define the \textit{path-length characteristic function}
\begin{equation*}
  \mathcal{X}^l_\nu(s) = \begin{cases}
    1, & s_\nu \leq s < s_{\nu+1} \\
    0, & \mbox{otherwise}.
    \end{cases}
\end{equation*}
Then, the piecewise constant representations of the path absorption coefficient $\tilde{a}(s)$ and the effective source $g_n(s)$ from Section \ref{sec:asymptotic_sol} are
\begin{align*}
  g_n(s) &= \sum_{\nu=1}^{N-1}g_{n\nu}\mathcal{X}^l_\nu(s), \\
  \tilde{a}(s) &= \sum_{\nu=1}^{N-1}\tilde{a}_{\nu}\mathcal{X}^l_\nu(s).
\end{align*}

As the ray traverses the spatial grids, it crosses $N-2$ spatial grid edges.
Let the nondecreasing path lengths at which these crossings occur be denoted by
$\left\{s_\nu\right\}_{\nu=1}^{N}$, with the convention $s_1=0$ and $s_{N}=\tilde{s}$.
$\{s_\nu\}$ is not strictly increasing if the ray directly intersects a grid corner,
which means that multiple edges are traversed at the same path length.
Hence, for $\nu=1,\ldots,N-1$, the path lengths through each grid cell are
\begin{equation*}
  ds_\nu = s_{\nu+1} - s_\nu.
\end{equation*}
Given $s$, the index next edge crossing occurs at
\begin{equation*}
  \hat{\nu}(s) = \min\left\{ \nu \in \{1,\ldots,N\} : s_\nu>s \right\},
\end{equation*}
and the path length between $s$ and the next edge crossing is
\begin{equation*}
  \tilde{d}(s) = s_{\hat{\nu}(s)}-s.
\end{equation*}
Then, evaluating \eqref{eqn:asymptotics_soln_n} at $s=\tilde{s}$ is calculated as
\begin{align*}
  u_n(\tilde{s}) &= \int_0^{\tilde{s}}g_n(s')\exp\left( -\int_{s''}^{s'}\tilde{a}(s'')\,ds'' \right)\, ds' \\
  &= \int_0^{s_N} \sum_{\nu=1}^{N-1}g_{n\nu}\mathcal{X}^l_\nu(s') \exp\left( -\int_{s''}^{s'}\sum_{j=1}^{N-1}\tilde{a}_{j}\mathcal{X}^l_j(s'')\,ds'' \right)\, ds' \\
  &= \sum_{\nu=1}^{N-1}g_{n\nu}\int_0^{s_N} \mathcal{X}^l_\nu(s') \exp\left( -\sum_{j=1}^{N-1}\tilde{a}_{j}\int_{s''}^{s'}\mathcal{X}^l_j(s'')\,ds'' \right)\, ds' \\
  &= \sum_{\nu=1}^{N-1}g_{n\nu}\int_{s_\nu}^{s_{\nu+1}}  \exp\left(-\tilde{a}_{\hat{\nu}(s')-1}\tilde{d}(s') -\sum_{j=\hat{\nu}(s')}^{N-1}\tilde{a}_{j}ds_j\right)\, ds' \\
  &= \sum_{\nu=1}^{N-1}g_{n\nu}\int_{s_\nu}^{s_{\nu+1}}  \exp\left(-\tilde{a}_{\nu}(s_{\nu+1}-s') -\sum_{j=\nu+1}^{N-1}\tilde{a}_{j}ds_j\right)\, ds'.
\end{align*}
This integral is made straightforward by setting
\begin{equation*}
  b_\nu = -\tilde{a}_{\nu}s_{\nu+1} - \sum_{j=\nu+1}^{N-1}\tilde{a}_{j}ds_j,
\end{equation*}
which yields
\begin{align*}
  u_n(\tilde{s}) &= \sum_{\nu=1}^{N-1}g_{n\nu}\int_{s_\nu}^{s_{\nu+1}}  \exp\left(\tilde{a}_{\nu}s' + b_\nu\right)\, ds' \\
                 &= \sum_{\nu=1}^{N-1}g_{n\nu}e^{b_\nu}\int_{s_\nu}^{s_{\nu+1}}  \exp\left(\tilde{a}_{\nu}s'\right) ds'.
\end{align*}
Define the intermediate variable
\begin{align*}
  d_\nu &= \int_{s_\nu}^{s_{\nu+1}}  \exp\left(\tilde{a}_{\nu}s'\right)\, ds' \\
    &= \begin{cases}
    ds_\nu, & \tilde{a} = 0 \\
      \left( \exp(\tilde{a}_\nu s_{\nu+1}) - \exp(\tilde{a}_\nu s_\nu) \right)/\tilde{a}_\nu, & \mbox{otherwise},
    \end{cases}
\end{align*}
which permits the simple formula
\begin{equation}
  u_n(\tilde{s}) = \sum_{\nu=1}^{N-1} g_{n\nu}d_\nu e^{b_\nu}.
  \label{eqn:discrete_ray_integral}
\end{equation}

\section{Finite Difference}
While the asymptotic solution is valid in case of low scattering, a more general solution is obtained via finite difference, whereby the derivatives and integrals in the integro-partial differential equation are discretized to differences and sums and evaluated at each grid cell in order to construct a linear system of equations whose solution approximates that of the analytical equation.
The price of a general solution, however, is greatly increased computational cost, both in terms of memory and CPU usage.
Not only is it necessary to store a potentially huge sparse matrix in memory, but its construction can take enormous amounts of time.
Nonetheless, it is useful to have access to a general numerical solution, at least for verification of approximations if not for direct use in applications.

\subsection{Discretization}

For the spatial interior of the domain, we use the second order central difference formula (CD2) to approximate the derivatives, which is
\begin{equation*}
    \tag{CD2}
    f'(x) = \frac{f(x+dx)-f(x-dx)}{2dx} + \mathcal{O}(dx^3).
\end{equation*}

When applying the PDE on the upper or lower boundary, we use the forward and backward difference (FD2 and BD2) formulas respectively.
Omitting $\mathcal{O}(dx^3)$, we have
\begin{equation*}
    \tag{FD2}
    \label{eq:FD2}
    f'(x) = \frac{-3f(x)+4f(x+dx)-f(x+2dx)}{2dx}
\end{equation*}
\begin{equation*}
    \tag{BD2}
    \label{eq:BD2}
    f'(x) = \frac{3f(x)-4f(x-dx)+f(x-2dx)}{2dx}
\end{equation*}

For the upper and lower boundaries, we need an asymmetric finite difference
method.
In general, the Taylor Series of a function $f$ about $x$ is
\begin{equation*}
  f'(x+\varepsilon) = \sum_{n=1}^\infty \frac{f^{(n)}(x)}{n!} \varepsilon^n \\
\end{equation*}

Truncating after the first few terms, we have
\begin{equation}
  \label{eqn:afd1}
  f'(x+\varepsilon)  = f(x) + f'(x)\varepsilon + \frac{f''(x)}{2}\varepsilon^2 + \mathcal{O}(\varepsilon^3)
\end{equation}

Similarly, replacing $\varepsilon$ with $-\varepsilon/2$ we have
\begin{equation}
  \label{eqn:afd2}
  f'(x-\frac{\varepsilon}{2}) = f(x) - \frac{f'(x)\varepsilon}{2} + \frac{f''(x)\varepsilon^2}{8} + \mathcal{O}(\varepsilon^3).
\end{equation}

Rearranging \eqref{eqn:afd1} produces
\begin{equation}
  \label{eqn:afd3}
  f''(x)\varepsilon^2 = 2f(x+\varepsilon) - 2f(x) - 2f'(x)\varepsilon + \mathcal{O}(\varepsilon^3)
\end{equation}

Combining \eqref{eqn:afd2} with \eqref{eqn:afd3} gives
\begin{align*}
  \varepsilon f'(x) &= 2f(x) - 2f(x-\frac{\varepsilon}{2}) + f''(x)\frac{\varepsilon^2}{8} + \mathcal{O}(\varepsilon^3) \\
                    &= 2f(x) - 2f(x-\frac{\varepsilon}{2}) + \frac{f(x+\varepsilon)}{4} - \frac{f(x)}{4} - \frac{f'(x)\varepsilon}{4} + \mathcal{O}(\varepsilon^3) \\
                    &= \frac{4}{5}\left( 2f(x)-2f(x-\frac{\varepsilon}{2}) + \frac{f(x+\varepsilon)}{4} - \frac{f(x)}{4} \right) + \mathcal{O}(\varepsilon^3)
\end{align*}

Then, dividing by $\varepsilon$ gives
\begin{equation*}
  f'(x) = \frac{-8f(x-\frac{\varepsilon}{2}) + 7f(x) + f(x+\varepsilon)}{5\varepsilon} + \mathcal{O}(\varepsilon^2)
\end{equation*}

Similarly, substituting $\varepsilon \to -\varepsilon$, we have 
\begin{equation*}
  f'(x) = \frac{- f(x-\varepsilon) - 7f(x) + 8f(x+\frac{\varepsilon}{2})}{5\varepsilon} + \mathcal{O}(\varepsilon^2)
\end{equation*}


\subsection{Difference Equations}
\label{sec:difference_equations}

%TODO: Periodic $x,y$

In general, we have

\begin{equation*}
  \vec{\omega} \cdot \nabla L_p = -(a+b) L_p + \sum_{p'=1}^{n_{\vec{\omega}}} \beta_{pp'}L_{p'}.
\end{equation*}

Then,
\begin{equation*}
  \vec{\omega} \cdot \nabla L_p + (a+b(1-\beta_{pp'}))L_p - \sum_{p'=1}^{n_{\vec{\omega}}} \beta_{pp'} L_{p'} = 0
\end{equation*}

Interior:
\begin{equation*}
  \begin{aligned}
    0 &= \frac{L_{i+1,jkp}-L_{i-1,jkp}}{2dx}\sin\hat{\phi}_p\cos\hat{\theta}_p \\
    &+ \frac{L_{i,j+1,kp}-L_{i,j-1,kp}}{2dy}\sin\hat{\phi}_p\sin\hat{\theta}_p \\
    &+ \frac{L_{ij,k+1,p}-L_{ij,k-1,p}}{2dz}\cos\hat{\phi}_p \\
    &+ (a_{ijk}+b(1-\beta_{pp'}))L_{ijkp}  - \sum_{p'=1}^{n_{\vec{\omega}}} \beta_{pp'} L_{ijkp'}
  \end{aligned}
\end{equation*}

Surface downwelling (BC):
\begin{equation*}
  \begin{aligned}
    0 &= \frac{L_{i+1,jkp}-L_{i-1,jkp}}{2dx}\sin\hat{\phi}_p\cos\hat{\theta}_p \\
    &+ \frac{L_{i,j+1,kp}-L_{i,j-1,kp}}{2dy}\sin\hat{\phi}_p\sin\hat{\theta}_p \\
    &+ \frac{-8f_p + 7L_{ijkp} + L_{ij,k+1,p}}{5dz}\cos\hat{\phi}_p \\
    &+ (a_{ijk}+b(1-\beta_{pp'}))L_{ijkp} \\
    &- \sum_{p'=1}^{n_{\vec{\omega}}} \beta_{pp'} L_{ijkp'}.
  \end{aligned}
\end{equation*}

Combining $L_{ijkp}$ terms on the left and moving the boundary condition to the
right gives

\begin{equation*}
  \begin{aligned}
    &\frac{L_{i+1,jkp}-L_{i-1,jkp}}{2dx}\sin\hat{\phi}_p\cos\hat{\theta}_p \\
    + &\frac{L_{i,j+1,kp}-L_{i,j-1,kp}}{2dy}\sin\hat{\phi}_p\sin\hat{\theta}_p \\
    + &\frac{L_{ij,k+1,p}}{5dz}\cos\hat{\phi}_p \\
    + &(a_{ijk}+b(1-\beta_{pp'}) + \frac{7}{5dz} \cos\hat{\phi}_p)L_{ijkp} \\
    - &\sum_{p'=1}^{n_{\vec{\omega}}} \beta_{pp'} L_{ijkp'} = \frac{8f_p}{5dz} \cos\hat{\phi}_p.
  \end{aligned}
\end{equation*}

Likewise for the bottom boundary condition, we have

\begin{equation*}
  \begin{aligned}
    0 &= \frac{L_{i+1,jkp}-L_{i-1,jkp}}{2dx}\sin\hat{\phi}_p\cos\hat{\theta}_p \\
    &+ \frac{L_{i,j+1,kp}-L_{i,j-1,kp}}{2dy}\sin\hat{\phi}_p\sin\hat{\theta}_p \\
    &- \frac{L_{ij,k-1,p}}{5dz}\cos\hat{\phi}_p \\
    &+ (a_{ijk}+b(1-\beta_{pp'}) - \frac{7}{5dz}\cos\hat{\phi}_p)L_{ijkp} \\
    &- \sum_{p'=1}^{n_{\vec{\omega}}} \beta_{pp'} L_{ijkp'}.
  \end{aligned}
\end{equation*}

Now, for upwelling light at the first depth layer (non-BC), we apply FD2.
\begin{equation*}
  \begin{aligned}
    0 &= \frac{L_{i+1,jkp}-L_{i-1,jkp}}{2dx}\sin\hat{\phi}_p\cos\hat{\theta}_p \\
    &+ \frac{L_{i,j+1,kp}-L_{i,j-1,kp}}{2dy}\sin\hat{\phi}_p\sin\hat{\theta}_p \\
    &+ \frac{-3L_{ijkp} + 4L_{ij,k+1,p} - L_{ij,k+2,p}}{2dz}\cos\hat{\phi}_p \\
    &+ (a_{ijk}+b(1-\beta_{pp'}))L_{ijkp} \\
    &- \sum_{p'=1}^{n_{\vec{\omega}}} \beta_{pp'} L_{ijkp'}.
  \end{aligned}
\end{equation*}

Grouping $L_{ijkp}$ terms gives
\begin{equation*}
  \begin{aligned}
    0 &= \frac{L_{i+1,jkp}-L_{i-1,jkp}}{2dx}\sin\hat{\phi}_p\cos\hat{\theta}_p \\
    &+ \frac{L_{i,j+1,kp}-L_{i,j-1,kp}}{2dy}\sin\hat{\phi}_p\sin\hat{\theta}_p \\
    &+ \frac{4L_{ij,k+1,p} - L_{ij,k+2,p}}{2dz}\cos\hat{\phi}_p \\
    &+ \left(a_{ijk}+b(1-\beta_{pp'}) - 3\frac{\cos\hat\phi_p}{2dz} \right)L_{ijkp} \\
    &- \sum_{p'=1}^{n_{\vec{\omega}}} \beta_{pp'} L_{ijkp'}.
  \end{aligned}
\end{equation*}

Similarly, for downwelling light at the lowest depth layer, we have
\begin{equation*}
  \begin{aligned}
    0 &= \frac{L_{i+1,jkp}-L_{i-1,jkp}}{2dx}\sin\hat{\phi}_p\cos\hat{\theta}_p \\
    &+ \frac{L_{i,j+1,kp}-L_{i,j-1,kp}}{2dy}\sin\hat{\phi}_p\sin\hat{\theta}_p \\
    &+ \frac{-4L_{ij,k-1,p} + L_{ij,k-2,p}}{2dz}\cos\hat{\phi}_p \\
    &+ \left(a_{ijk}+b(1-\beta_{pp'}) + 3\frac{\cos\hat\phi_p}{2dz} \right)L_{ijkp} \\
    &- \sum_{p'=1}^{n_{\vec{\omega}}} \beta_{pp'} L_{ijkp'}
  \end{aligned}
\end{equation*}

\subsection{Structure of Linear System}
For each spatial-angular grid cell, one of the above equations is applied.
The equation applied at each grid cell involves adjacent radiance values due to the discretized derivatives.
Thus, a coupled system of linear equations is produced, which can be written as a sparse matrix equation, $Ax=b$.
In the coefficient matrix $A$, each row is asociated with the grid cell at which the discretized equation was evaluated.
Each column is the coefficient of the radiance at a particular spatial-angular grid cell.
In principle the order of the equations, i.e., the order of the rows and columns of the coefficient matrix, is not important
so long as consistency is maintained with the solution vector and right-hand side.
In general, some procedure is necessary for constructing an ordered list of the multidimensional grid cells.
One option, employed here, is to use a block structure where dimensions are nested within one another.
An ordering for the dimensions is chosen, from outermost to innermost.
Adjacent rows and columns in the matrix are associated with adjacent grid cells in the innermost dimension,
adjacent blocks of the innermost dimension are adjacent in the second innermost dimension, etc.

In the numerical implementation of this model, we choose the order of dimensions to be $\vec{\omega}, z, y, x$, with $\vec{\omega}$ being the outermost and $x$ being the innermost.
Recall that $\theta$ and $\phi$ are already combined, both indexed by $p$, as discussed in Section \ref{sec:grid} and Appendix \ref{chap:grid_details}.
This particular ordering is chosen for ease of programming in terms of deciding which of the equations from Section \ref{sec:difference_equations} to apply.
Since the choice of equation does not depend on $x$ or $y$, they are the outermost.
Then, the surface and bottom $z$ values have to be considered separately from the rest.
And within the surface and bottom depth layers, there are further cases depepending on whether the light is upwelling or downwelling.
Hence, the chosen ordering follows somewhat naturally from the boundary conditions.

Then, the discretized equation applied to $(x_i, y_j, z_k, \omega_p)$ is stored in row
\begin{equation*}
  r_{ijkp} = p + \nomega (k-1) + \nomega n_z (j-1) + \nomega n_z n_y (i-1).
\end{equation*}
Since the same ordering is used for rows and columns of the coefficient matrix $A$, $L_{ijkp}$ is located at position $r_{ijkp}$ of the solution vector $x$,
and the right-hand side associated with that grid cell, if any, is also stored at position $r_{ijkp}$ of the right-hand side vector $b$.

Also relevant is the total size of the system and of the sparse matrices necessary to store.
The sizes of $A$, $x$, and $b$ are the number of grid cells, which is just $n_xn_yn_z\nomega$.
Most of these elements, though, are zero since derivatives only involve adjacent spatial grid cells and the scattering integral only involves angles within a single spatial grid cell.
Therefore, by saving only the locations and values of nonzero elements in the coefficient matrix, a considerable amount of storage space is saved.
Table \ref{tab:nonzero} shows a breakdown of the number of distinct radiance values involved in each application of the discretized equations from Section \ref{sec:difference_equations}, as well as the number of times that each of the equations appears in the matrix.

\begin{table}[H]
  \centering
  \begin{tabular}{p{\linewidth/3}p{\linewidth/3}p{\linewidth/3}}
    \toprule
    \textbf{Derivative case} & \textbf{\# nonzero/row} & \textbf{\# of rows} \\
    \midrule
    interior & $\nomega+6$ & $n_xn_y(n_z-2)\nomega$ \\
    surface downwelling & $\nomega+5$ & $n_xn_y\nomega/2$ \\
    bottom upwelling & $\nomega+5$ & $n_xn_y\nomega/2$ \\
    surface upwelling & $\nomega+6$ & $n_xn_y\nomega/2$ \\
    bottom downwelling & $\nomega+6$ & $n_xn_y\nomega/2$ \\
  \end{tabular}
  \caption{Breakdown of nonzero matrix elements by derivative case}
  \label{tab:nonzero}
\end{table}

By multiplying the first column of Table \ref{tab:nonzero} by the second and summing over the rows, the total number of nonzero matrix elements is calculated to be
\begin{align*}
  N_A &= (\nomega+6) \cdot n_xn_y(n_z-2)\nomega \\
    &+   (\nomega+5) \cdot n_xn_y\nomega
    +   (\nomega+6) \cdot n_xn_y\nomega \\
  &= n_x n_y n_z \left[(\nomega+6)(n_z-2+1)+\nomega+5 \right] \\
  &= n_x n_y n_z \left[(\nomega+6)(n_z-1)+\nomega+5 \right] \\
  &=  n_x n_y n_z \left[n\omega n_z -\nomega + 6n_z - 6 + \nomega + 5 \right] \\
  &=  n_x n_y n_z \left[n_z(\nomega+6)-1\right]
\end{align*}
Also, note that $b$ only has nonzero entries for the downwelling surface grid cells, of which there are $n_x n_y \nomega/2$.

\subsection{Iterative Solution}
For small systems of equations, direct methods such as Gaussian elimination, QR factorization, and singular value decomposition can be used.
However, when the coefficient matrix becomes very large, the memory required to carry out a direct solution is prohibitively large, iterative solvers must be used.
Many such solvers are available, including GMRES\cite{saad_gmres:_1985}, LGMRES\cite{baker_technique_2005}, IDR\cite{sonneveld_idrs:_2008}, and BI-CGSTAB\cite{van_der_vorst_bi-cgstab:_1992}.
In our case, GMRES is used.

